%%%%%%%%%%%%%%%%%%%%%%%%%%%%%%%%%%%%%%%%%%%%%%%%%%%%%%%%%%%%%%%%%%%%%%%%%%%%%%%
%
% technical_report.tex
%
%                  Daniel Holmes, Jonathan Gerrand (2 October 2014)
%
%                    ELEN3009 Software Project Technical Report
%
%%%%%%%%%%%%%%%%%%%%%%%%%%%%%%%%%%%%%%%%%%%%%%%%%%%%%%%%%%%%%%%%%%%%%%%%%%%%%%%%

\documentclass[10pt,twocolumn]{witseiepaper}

%
% All KJN's macros and goodies (some shameless borrowing from SPL)
\usepackage{KJN}
\usepackage{array}

%
% PDF Info
%
\ifpdf
\pdfinfo{
/Title (SOFTWARE PROJECT TECHNICAL REPORT)
/Author (Daniel G. Holmes Jonathan D. Gerrand)
/CreationDate (D:201410100253)
/ModDate (D:201404241400)
/Subject (Template Report)
/Keywords (Report, Template)
}
\fi

%%%%%%%%%%%%%%%%%%%%%%%%%%%%%%%%%%%%%%%%%%%%%%%%%%%%%%%%%%%%%%%%%%%%%%%%%%%%%%%
\begin{document}


\title{SOFTWARE PROJECT TECHNICAL REPORT}

\author{Daniel G. Holmes 551240, Jonathan D. Gerrand 349361
\thanks{School of Electrical \& Information Engineering, University of the
Witwatersrand, Private Bag 3, 2050, Johannesburg, South Africa}
}

%%%%%%%%%%%%%%%%%%%%%%%%%%%%%%%%%%%%%%%%%%%%%%%%%%%%%%%%%%%%%%%%%%%%%%%%%%%%%%%
%
\abstract{Abstract}

\keywords{SFML 2.1, game development, MVP}


\maketitle
\thispagestyle{empty}\pagestyle{empty}


%%%%%%%%%%%%%%%%%%%%%%%%%%%%%%%%%%%%%%%%%%%%%%%%%%%%%%%%%%%%%%%%%%%%%%%%%%%%%%%
%
\section{INTRODUCTION}% rough points
Problem understanding
- Modular code
- Orthogonal
- Object-oriented decomposition
- Separation of layers (MVP, MVC)
- Illustrate design cycle of software
- Iterative design process
- Unit testing approach (regression testing)

%%%%%%%%%%%%%%%%%%%%%%%%%%%%%%%%%%%%%%%%%%%%%%%%%%%%%%%%%%%%%%%%%%%%%%%%%%%%%%%
%
\section{REQUIREMENTS AND CONSTRAINTS}% first iteration
The game is required to be based on the 2 player tank battle Atari game "Combat". Certain basic functionality is required to be met. Each player should be able to control their tanks from the computer keyboard. Tanks should have the ability to fire missiles and lay mines, both of which destroy either players tanks. Movement of tanks and firing of tanks should be done both horizontally and vertically. Game play should take place in a maze consisting of barriers which block tank movement. Once either of the tanks have been destroyed, the game should end.

The game software is required to be implemented in ANSI/ISO C++. The usage of the SFML 2.1 library is also required. The software development life-cycle requires the usage of unit tests using the google-test framework. Unit tests are required at the end of each design iteration of the game.

The game is constrained to run on the Windows platform. No other libraries that are built on SFML may be used. With respect to game graphics, no OpenGL may be used. Also, the maximum resolution of the game window should not exceed 1600x900. 

It is assumed that the computer running the game will meet the minimum system requirements for Windows 7. It is also assumed that SFML functions have been tested and work as expected.

%%%%%%%%%%%%%%%%%%%%%%%%%%%%%%%%%%%%%%%%%%%%%%%%%%%%%%%%%%%%%%%%%%%%%%%%%%%%%%%
%
\section{BACKGROUND} % first iteration
Games are classic examples of highly interactive Graphical User Interface (GUI) applications. Over the years, many different GUI architectures have emerged. One such architecture is Model View Controller (MVC). The main idea of MVC is a separation of the domain objects and GUI elements. Domain objects model real world perceptions, whereas the GUI elements are what the user sees on the screen. Simply put, the model objects have no knowledge of the UI. The controller role within this architecture is to process user input. This either directly or indirectly would modify the model, or the domain objects. Once data changes to the model are complete, information from the model is pushed to the presenter and the users view is updated accordingly [martin-fowler].

One of the benefits of the MVC architecture is the fact that there is a distinct separation of program layers. These include presenter, logic and data layers. The usage of MVC ensures orthogonality between these program layers. Many games these days are available on multiple platforms. This necessitates flexibility in terms of porting over a game to another platform. The game would require specific libraries; most commonly the display libraries. A game could be ported easily if there is loose coupling between program layers. For example, a game developed initially on a PC would have a different presentation layer requirements compared to mobile game. If the separation of layers is successfully achieved, a considerable less amount of effort will be required to port the game. A game in which the port is almost indistinguishable from the original is know as an Arcade Perfect Port [tvtropes].

The Agile methodology to project management is an alternative to traditional project management. The focus of Agile is on developing a product that is potentially shippable. This requires a iterative and incremental approach to product development. It facilitates communication between the development team in which the project requirements and design are continuously revisited. This allows for flexibility as problems in the product development life-cycle can be solved sooner, particularly if that problem relates to a change in requirements [agile].

This methodology is well suited to the development of software. The focus is on having software that can be shipped to users sooner, so that early feedback can be given. Overall, the usage of Agile helps developers put together the right product in as little time as possible.

%%%%%%%%%%%%%%%%%%%%%%%%%%%%%%%%%%%%%%%%%%%%%%%%%%%%%%%%%%%%%%%%%%%%%%%%%%%%%%%
%
\section{DESIGN OVERVIEW} % first iteration
The chosen design is based off the MVC GUI model.the main Figure 1 shows the adapted version of MVC know as passive MVC used as a high level design for the game. A typical MVC design would include the model actually updating the view. With passive MVC, the display polls the model for changes. The polling methodology is suited to be used in the way the main program loop executes, where the display is updated at the end of the loop. 

Another modification inherit in the design is the connection between the controller and the data. Typically, MVC would have the controller simply updating the data. Figure 1 shows information moves in both directions between the controller and data. Model data is not only updated, but it is also retrieved by the controller to be used in logical operations. Another reason for this particular modification of MVC is due to the use of managers. During a particular management cycle, information from the model is retrieved, processed and then the model is updated accordingly. Only once all the data has been processed by the managers does the display get updated.  

%Perhpas include a game summary of what actually happens in the game

%overview figure will be included here

%%%%%%%%%%%%%%%%%%%%%%%%%%%%%%%%%%%%%%%%%%%%%%%%%%%%%%%%%%%%%%%%%%%%%%%%%%%%%%%
%
\section{CONCEPTUAL MODEL} % first iteration

The following section presents the conceptual model of the game in terms of the MVC architecture.

\subsection{Controller}
The controller module of the design is based on management cycles or ticks. These cycles occur within the main program loop. Management tasks are divided between different managers, each of which have specific management tasks which introduces single responsibility classes. These managers perform their tasks based on user input. Information regarding this user input, as well as other temporary main loop data are accessed by the managers. Each manager only has access to data pertinent to it's functions. These functions include create, read, update and delete (CRUD) operations.

The usage of managers also exploits the principle polymorphism. The fundamental task of a manager relates to an inherited property of the objects which the manager is managing. For example, all objects with the inherited property "Movable" are accessible by the Move Manager. An object may have many inherited properties. As such, the management of that object is taken care of by the relevant managers, purely because that object inherits certain properties. A new manager may be added based on the inherited properties of objects, or as per the abstraction requirements.

This type of approach in the controller module ensures that the CRUD operations are dependant upon the characteristics of game objects. This also means that the controller is able to meet the processing requirements of the model in a modular way. If there is a problem with a particular aspect of the program, one simply needs to search in the relevant manager. If a new manager is added to the controller module, it has impact on the other managers. This ensures that if new game functionality is added, the other functionality is not affected. Fundamentally, controller modules i.e. the manager, are agnostic to other management processes.

Another component of the controller module includes the user input. Due to the nature of the game, a user does not input information via the User Interface (UI), as is the case with the Model View Presenter (MVP) architecture. Rather, the information is received from the user via the keyboard. This is another reason why the MVC model is used in the game design. User input can go directly to the controller module. Thus, managers can act on user commands without involving the view module.

\subsection{Model}
All game entities (models) exist within the model module. Each model inherits properties from certain Abstract Based Classes. As such, a model may inherit multiple properties as far as the level of model abstraction requires. All game models inherit the base properties "Deletable" and "Drawable". This is so that all models can be deleted or removed from the game world as required. Also, all game entities are drawn at some stage in the program cycle. An example of a game model which inherits multiple properties is a tank. It inherits the properties "Movable", "Colliable" and "Trackable". Each of these properties gives an indication of the functionality of the tank model. Also, the managers (move, collision and tracking) that have access to the tank model data are evident.

In line with MVC and the separation of layers principles, the models do not perform any decision making. Most of the functionality of the models include setter functions, or "tell, don't ask" functions.

\subsection{View}
The view module contains all the functions required for drawing game entities within the game window (UI). In line with passive MVC, the data from the model is not pushed to the view. Rather, the view polls for changes in data. Only information required for drawing is retrieved. 

This methodology further emphasises the point of separating the layers of the program. If a view is required on a different platform, then the existing view can be replaced without any impact on the model or controller modules. The controller module is also loosely coupled to the view. Overall, the separation of the view makes the game potentially portable. Interface of the view classes can remain mostly unchanged while the implementation can be changed accordingly.

%%%%%%%%%%%%%%%%%%%%%%%%%%%%%%%%%%%%%%%%%%%%%%%%%%%%%%%%%%%%%%%%%%%%%%%%%%%%%%%
%
\section{CLASS RESPONSIBILITIES} % first iteration

%This section presents responsibilities of each major class.
%(perhaps include a table that shows all classes and in what layer they are in)

%\subsection{Abstract Based Classes}
% maybe...?

\subsection{Game Class} 
This class is responsible for running the main game loop and exists within main. Within this loop, the Game class receives user input, runs the managers and creates new entities. The creation of new entities depends upon user input, such as firing a missile or laying a mine. User input is received via the Keyboard class.

The game class is also responsible for getting map data. The map is read from a text file in which certain text symbols represent a game entity. Upon initial game execution, the map text file is read and the starting positions of game entities are set accordingly. 

It is within the Game class that game entities are added and prepared for managers to work with. This is where the polymorphic nature of game entities is used. All managers exist within the this class. It is from here that all managers are called to manage. The Display class is also used in the Game class. However, the Display object does not exist within the Game class. Display functions are called in the Game class by reference. 

\subsection{Display Class}
This class is responsible for drawing the UI. This includes game sprites and textures. It is also responsible for loading sprite images and textures. The object of this class exists in main along with the Game class. This ensures a limitation of the control that the Game class has over the Display object. Also, it provides a separation between the controller and view, as per MVC.

\subsection{Manager Classes}
Table 1 summarises the main responsibilities of each of the game managers. Although the draw manager is included in Table 1, it does not form part of the controller module. The draw manager does not manipulate data as most of the other managers do. Instead, the draw manager prepares data for drawing by extracting it from the models. It then passes this data to the view for drawing. For this reason, the draw manager forms part of the view module of the program. 
%Manager abstract based class?
\begin{table}[h]
	\centering
	\caption{Summary of the main responsibilities of the game managers}
	\begin{tabular}{|>{\centering\arraybackslash}m{2cm}|>{\centering\arraybackslash}m{4.5cm}|}
		\hline 
		\textbf{Manager} & \textbf{Responsibility} \\ 
		\hline 
		Move & Instructing game entities to move based on user input \\ 
		\hline 
		Collision & Determining if game entities have and with what game 								entities have collided with \\ 
		\hline 
		Tracking & Keeps track of player and turret orientation \\ 
		\hline 
		Turret & Rotates all game turrets \\ 
		\hline 
		Game State & Manage all game timers \\ 
		\hline 
		Destruction & Deletes destroyed game entities \\ 
		\hline 
		Draw & Prepares data of entities to be drawn \\ 
		\hline 
	\end{tabular} 
\end{table}

\subsection{Data Classes}
\subsubsection{Model Classes} The model classes include all game entity classes. Their responsibilities are limited to holding information about a given game entity. An important aspect of these classes is that all their functions are inherited virtual functions. None of these classes have private functions. This ensures that these classes have very little flexibility in terms of what they are capable of. They exits to be not to act, but be acted upon.

\subsubsection{Game data} Apart from model classes, the game also includes a GameManagementData class. This class inherits from ActionData and GameStateData classes. These classes have no specific responsibilities other than to provide information for the managers. Managers will only receive either, one or both of the parent classes of GameManagementData. This is so that a manager only receives information required to perform its management tasks.

\subsection{Usage of SFML} % incomplete first iteration
As per the game requirement, SFML 2.1 is used. The following subsection discusses the usage SMFL libraries throughout the code. Justification relating to why SFML was not used where is otherwise may have been used is also given.

\subsubsection{Dependant Classes}  Attempts have been made to decouple SFML from the main components of the design. No SFML is included in the model module. It has been included in the view and controller modules. However, classes have been design in such a way that these modules are only linked to SFML via interfaces. The only instances where SFML is used is for capturing user input and drawing sprites in the game window.

The Keyboard class, which makes use of the SFML keyboard functionality, is included in the controller. Due to the fact that the controller uses this class via an interface, the controller is ignorant of the library used in the keyboard class. The same principle applies to the Display class, except with respect to its use in the view module. 

\subsubsection{Draw Manager} Although the draw manager forms part of the view, it is not dependant upon any SFML libraries. Thus, the orthogonality of the code with respect to the drawing of entities is maintained. It is only the Display class that is dependant upon SFML drawing functionality.

\subsubsection{Collision Detection} One of the key functions in most games is collision detection. SFML does include functionality for collision detection. However, this was not used. The reason for this is not only as an attempt to decouple SFML. Rather, the replacement of existing collision detection functionality is for the purpose of code reuse. This point is discussed further as part of the next subsection.

\subsection{GeometryEngine}
Collision detection is included in the GeometryEngine class. This is a class that assists several managers with geometrical related functions. Some of the functions used for collision detection are also used for determining if a tank is in direct line of fire of a given turret. These functions would need to have been coded anyway. 

Thus, code is reused in two different major decision making functions. This is one of the reasons why SFML collision detection was replaced. 

Another responsibility of the GeometryEngine was to handle logic for tank missile reflection. This is implemented as part of a major feature which does not form part of the basic functionality requirements.

%refer to UML relationship diagram

%refer to UML class diagram with Game, DrawManager and Display classes


%%%%%%%%%%%%%%%%%%%%%%%%%%%%%%%%%%%%%%%%%%%%%%%%%%%%%%%%%%%%%%%%%%%%%%%%%%%%%%%
%
%\section{DYNAMIC BEHAVIOUR} 
%refer to the logic diagram in appendix which shows proram logical flow

%%%%%%%%%%%%%%%%%%%%%%%%%%%%%%%%%%%%%%%%%%%%%%%%%%%%%%%%%%%%%%%%%%%%%%%%%%%%%%%
%
\section{MEMORY MANAGEMENT} % first iteration

Due to the extensive use of polymorphism, memory management forms a crucial part of the design. A game entity is created as a shared pointer and exists within the Game class object. In order to take advantage of polymorphism, each game entity pointer is added to a vector container. To ensure that correct memory management practices are adhered to, these entity pointers are cast as weak pointers. 

These vectors thus contain weak pointers that are of a type that corresponds to an inherited property of game entities. For example, all collidable entities added to a vector of weak pointers of type "Collidable". It is from these vectors that managers can have access to relevant game entity objects.

Using weak pointers has several benefits in this case. Firstly, using weak pointers ensures that that the pointer and object get properly deleted. If the weak pointer gets deleted within a certain manager, all other pointers within other vectors will also be deleted. This ensures that the other managers do not try and work with pointer to objects that do not exist any more.

%perhpas need some more discussion on weak pointers

%perhaps it is worth mentioning the memory management in Disply with the map


%%%%%%%%%%%%%%%%%%%%%%%%%%%%%%%%%%%%%%%%%%%%%%%%%%%%%%%%%%%%%%%%%%%%%%%%%%%%%%%
%
\section{TESTING} % first iteration, incomplete

\subsection{Unit Tests}
The Agile methodology was used for the development of the game. As such, testing occurred at multiple stages in the development life-cycle. This testing included normal gameplay testing, as well as unit tests. These tests took place at three different stages. Firstly, while doing exploratory use of SFML. Secondly, after basic functionality was implemented. Thirdly, once the final version of the game was completed.

This approach was taken to facilitate regressing testing. Using regressing testing ensures that added functionality does not break previously implemented functionality. As part of using an iterative approach, source code was revised and tested several times. Analysis of the unit tests not discussed in this report. Testing results are discussed in terms of the high level requirements for the game.

\subsection{Final Game Functionality}
All basic game functionality was implemented. In addition, two minor and two major features were implemented. The minor features include a scoreboard and game time-limit, as well as map editor functionality. One major feature includes the ability of tanks to move and fire and any direction, with missiles able to reflect off walls/barriers. The other major feature includes the existence of gun turrets which target both players. These turrets form part of the barriers and can be destroyed by either player. Players are only targeted when the tanks are within a certain range of the turrets and are within the line of fire.

\subsection{Gameplay Testing}
Several bugs and performance issues were highlighted during normal gameplay testing. The most noticeable issue is that a framerate drop occurs when there are many sprites in the game window. 

One of the major bugs that exist in the game relate to the turrets and their firing at tanks. If the tank is behind the turret and at a certain rotation, the turret fires. After inspection of the source code, it was found that a certain function was not giving correct information. %unit tests for the get bounding box function?

%other gameplay testing?

%%%%%%%%%%%%%%%%%%%%%%%%%%%%%%%%%%%%%%%%%%%%%%%%%%%%%%%%%%%%%%%%%%%%%%%%%%%%%%%
%
\section{CRITICAL ANALYSIS} % rough

\subsection{Maintainability}

\subsection{The DRY Principle}
Should have used template functions more.

\subsection{Long Functions}

\subsection{Monolithic Classes}
Game class

\subsection{Tight Coupling of Classes}

\subsection{Effective use of Inheritance}
Barriers could not be blocked

\subsection{Use of Idiomatic C++}
Struct instead of classes

Tie back to what was observed in testing. The slow gameplay can be attributed to the order $ n^{2} $ calculations in collision detection. In this function, the bounding box of each entity is compared to the bounding box of every other entity. This results in many unnessary calculations.

Structs were used extensively, it would have been better if these were separated into classes. Have to always reference our structs file, would be better to reference the individual class. 

Talk about things that were not observed from testing, including: violated dry principle in certain places. Would be better to use templates.

There are several refused bequests, such as with a turret which is a movable, but cannot move forward or backward.

Talk about monolithic classes.

%%%%%%%%%%%%%%%%%%%%%%%%%%%%%%%%%%%%%%%%%%%%%%%%%%%%%%%%%%%%%%%%%%%%%%%%%%%%%%%
%
\section{RECOMMENDATIONS} % rough
Don't recommend changing the interface. Suggest improvements on the existing design, not on how to change the design.

To reduce the number of calculations for collision detections, there can be certain checks put in place to see if the entities are within a certain threshold of each other. The entities are not within the range of each other, then the collision detection algorithm does not need to be run.

Perhaps recommends a better design methodology, not too specific though.

Create more specific abstract based classes.
Have a map class.
Have a creation manager.

%%%%%%%%%%%%%%%%%%%%%%%%%%%%%%%%%%%%%%%%%%%%%%%%%%%%%%%%%%%%%%%%%%%%%%%%%%%%%%%
%
\section{CONCLUSION}


%%%
% Automatically balance the output of the last page
\balance
%%%

%%%%%%%%%%%%%%%%%%%%%%%%%%%%%%%%%%%%%%%%%%%%%%%%%%%%%%%%%%%%%%%%%%%%%%%%%%%%%%%
%
%\nocite{*}
\bibliographystyle{witseie}
\bibliography{references}

- MVP, MVC (martin fowler)
- Logic layer definition
- Games (Rayman)
- Textbooks?

\end{document}
